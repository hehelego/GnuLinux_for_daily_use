\documentclass[8pt,t]{beamer}

\usepackage{xeCJK}
\setCJKmainfont{Noto Serif CJK SC}

\usepackage{minted}
\setlength{\parindent}{2em}
\usepackage{amsmath}
\usepackage{amssymb}
\usepackage{graphicx}
\usepackage{xcolor}
\usepackage{ulem}
\newcommand{\passthrough}[1]{#1}
\usepackage{hyperref}
\hypersetup{
	colorlinks=true,
	linkcolor=cyan,
	filecolor=cyan,
	urlcolor=cyan,
	citecolor=cyan,
}

% metadata
\date{\today}
\title{一些我觉得有用的工具和方法}
\author{spinach/hehelego}
\institute{GeekPie@ShanghaiTech}

% beamer settings
\usetheme{Madrid}
\usecolortheme{seahorse}
\RequirePackage{geometry}
\setbeamersize{text margin left=.3cm}
\setbeamersize{text margin right=.3cm}
\setbeamerfont{footnote}{size=\tiny}

\begin{document}


\begin{frame}
	\titlepage{}
\end{frame}

\begin{frame}
	\frametitle{Sample}

	In this slide, some important text will be\footnote{footnote will be here}
	\alert{highlighted} because it's important.
	Please, don't abuse it.

	\begin{block}{Remark}
		Sample text
	\end{block}

	\begin{alertblock}{Important theorem}
		Sample text in red box
	\end{alertblock}

	\vfill

	\begin{examples}
		Sample text in green box. The title of the block is ``Examples''.
	\end{examples}
\end{frame}

\begin{frame}[c]
	\frametitle{声明}
	\begin{itemize}
		\item 如标题所言, 只是介绍一些我认为有用的, 我自己经常使用的工具.\\
		\item 如果您发现我介绍的某个工具/某种技术在您的平台没有官方支持.
		\item 如果您发现我口胡的某个理论有明显谬误.
		\item 如果您发现我的讲述方式过于不友好.
		\item 如果您发现我在浪费您的时间.
		\item $\cdots$
	\end{itemize}
	\pause{}
	请您务必及时联系我, 指出问题所在, 我会尽快进行修正\\
	因为我非常重视我的产出的质量, 尤其是公开于其他人的内容.\\
	\sout{欢迎来打爆我qwq...QQ,微信平台除外}
	\vfill{}
	\pause{}
	\begin{center}
		So, Let's get started\\
	\end{center}
\end{frame}


%%%%%%%%%%%%%%%%%%%%%%%%%%%%%%%%%%%%%%%%%%%%%%%%%%%%%%%%%%%%%%%%%%%%%%%%%%%%%%%%%%%%%%%%%%%%%%%%%%
%%%%%%%%%%%%%%%%%%%%%%%%%%%%%%%%%%%%%%%%%%%%%%%%%%%%%%%%%%%%%%%%%%%%%%%%%%%%%%%%%%%%%%%%%%%%%%%%%%

\begin{frame}
	\frametitle{fish shell + python + GNU coreutils}
	\begin{block}{这是什么}
		是工作环境的基础,
		是整个系统暴露给我们的接口, 通过它们进行各种操作.\\

		\begin{itemize}
			\item fish: \textcolor{red}{f}riendly \textcolor{red}{i}nteractive \textcolor{red}{sh}ell\\
			      \begin{itemize}
				      \item sematic colorizing (path, command, option, varibles)
				      \item intuitive syntax\&commands
				      \item suggestions\&completion
				      \item \emph{work out of the box}
			      \end{itemize}
			\item python: a \href{https://www.python.org/dev/peps/pep-0020/}{Beautiful,Explicit,Simple,Sparse} programming language\\
			      \href{https://www.python.org/}{that lets you work quickly and integrate systems more effectively}
			      \begin{itemize}
				      \item suitable for scripting (automatically perform tasks for human).
				      \item well-developed eco-system for nearly every field.
				      \item get support from powerful\&friendly python users everywhere.
			      \end{itemize}
			\item GNU coreutils: ls, rm, mkdir, cat, tee $\ldots$
		\end{itemize}
	\end{block}

	\begin{examples}
		这怎么用?
	\end{examples}
\end{frame}


\begin{frame}[fragile]
	\frametitle{trash-cli}

	\begin{alertblock}{terrible (but common) cases}
		\begin{minted}{fish}
rm ~/secret-donot-let-the-TAs-know/spinach-cs100-hw-source.tar.gz
rm ~ /secret-donot-let-the-TAs-know/spinach-cs101-hw-source.tar.gz
		\end{minted}
	\end{alertblock}

	\vfill

	\begin{examples}
		\begin{minted}{fish}
alias rm="trash"
		\end{minted}

		{\scriptsize{
			\begin{itemize}
				\item Delete a file (send to trash): \mint{fish}|trash path/to/file|
				\item List files in trash: \mint{fish}|trash-list|
				\item Restore file from trash: \mint{fish}|trash-restore|
				\item Empty trash: \mint{fish}|trash-empty|
				\item Empty trash, keeping files trashed less than $10$ days ago: \mint{fish}|trash-empty 10|
				\item Remove all files named \texttt{`foo'} from the trash: \mint{fish}|trash-rm foo|
				\item Remove all files with a given original location: \mint{fish}|trash-rm /absolute/path/to/file_or_directory|
			\end{itemize}
		}}
	\end{examples}

	futher: \textbf{btrfs} subvolumn snapshot
\end{frame}

\begin{frame}
	\frametitle{git: \sout{a version control software} the distributed version control system}
\end{frame}


\begin{frame}
	\frametitle{broot/ranger}
	\sout{explorere.exe in your terminal emulator}
	\begin{block}{what is broot/ranger}
		\begin{itemize}
			\item ranger: A simple vim-like TUI file manager
			\item broot: fuzzy search path + tree view + quick cd
		\end{itemize}
	\end{block}


	\vfill

	\begin{examples}
		效果如图所示.
	\end{examples}
\end{frame}

\begin{frame}
	\frametitle{fzf + rg/ag}
	\begin{block}{what is fzf, rg, ag}
		\begin{itemize}
			\item fzf: \textcolor{red}{F}uzzy \textcolor{red}{F}ile \textcolor{red}{F}inder
			\item rg: rip grep. Search for matched patterns.
			\item ag: the silver searcher. Code Searching tool.
		\end{itemize}
	\end{block}


	\vfill

	\begin{examples}
		效果如图所示.
	\end{examples}
\end{frame}


\begin{frame}
	\frametitle{(neo) vim\footnote{or emacs, xray} \sout{a editor} the editor!}

	\begin{block}{features}
		\begin{itemize}
			\item multiple editing modes (insert, normal, visual $\ldots$ up to 12 modes)
			\item customizability (builtin \texttt{vimscript/lua} VM + python/node.js/ruby RPC)
			\item lightweight
			\item strong community support
		\end{itemize}
	\end{block}

	\vfill
	\begin{alertblock}{No more MS visual studio code}
		\begin{itemize}
			\item Microsoft Visual Studio Code.  $(\times)$
			\item vscode Code-OSS $(\surd)$
		\end{itemize}
	\end{alertblock}
\end{frame}

%%%%%%%%%%%%%%%%%%%%%%%%%%%%%%%%%%%%%%%%%%%%%%%%%%%%%%%%%%%%%%%%%%%%%%%%%%%%%%%%%%%%%%%%%%%%%%%%%%
%%%%%%%%%%%%%%%%%%%%%%%%%%%%%%%%%%%%%%%%%%%%%%%%%%%%%%%%%%%%%%%%%%%%%%%%%%%%%%%%%%%%%%%%%%%%%%%%%%

\begin{frame}[c]
	\centering \huge {\textcolor{red}{Do not miss the KEY part of this lecture}}
\end{frame}

\begin{frame}
	\frametitle{TL;DR pages}
	\begin{block}{What is TL;DR pages}
		Simplified and community-driven man pages
		\begin{itemize}
			\item ``TL;DR'' stands for \emph{too long; don't read} or 太长不看 in Chinese.
			\item \href{https://tldr.sh/}{tldr.sh} is the official site for TL;DR pages project.
		\end{itemize}
	\end{block}


	\vfill
	\begin{examples}
		DFS
		\begin{enumerate}
			\item \mint{fish}|tldr systemd-timesyncd|
			\item \mint{fish}|man systemd-timesyncd|
			\item \mint{fish}|curl https://www.google.com/?q=systemd-timesyncd|
		\end{enumerate}
	\end{examples}
\end{frame}

\begin{frame}
	\frametitle{search engines}
	\begin{block}{effective searching tips}
		\begin{itemize}
			\item search by *\emph{keywords}*\\
				\sout{使用搜索引擎就是查字典, 你难道和字典讲自然语言吗}
			\item use the advanced filtering commands
			\item take advantages over vertical search and in-site search
		\end{itemize}
	\end{block}

	\vfill

	\begin{examples}
		呜呜, 我的kernel挂了. 你能帮帮我吗qwq.
	\end{examples}

	\vfill

	\begin{block}{Troubleshooting: how-to}
		\begin{enumerate}
			\item information gathering
			\item fix it your self?
			\item requesting for support
			\item hacking!
			\item \(\infty\) \emph{recording \& sharing}
		\end{enumerate}
	\end{block}

\end{frame}



%%%%%%%%%%%%%%%%%%%%%%%%%%%%%%%%%%%%%%%%%%%%%%%%%%%%%%%%%%%%%%%%%%%%%%%%%%%%%%%%%%%%%%%%%%%%%%%%%%
%%%%%%%%%%%%%%%%%%%%%%%%%%%%%%%%%%%%%%%%%%%%%%%%%%%%%%%%%%%%%%%%%%%%%%%%%%%%%%%%%%%%%%%%%%%%%%%%%%

\begin{frame}
	\frametitle{A few more things \sout{你们还是另请高明吧}}
	\begin{block}{(personally) recomendedd applications}
		\begin{itemize}
			\item i3wm, bspwm
			\item rofi
			\item tmux
			\item graphviz
			\item surfing keys / vimnium
		\end{itemize}
	\end{block}

	\begin{block}{resources}
		\begin{itemize}
			\item wikipedia
			\item stackexchange
			\item github
			\item arch linux community wiki
		\end{itemize}
	\end{block}

	\begin{itemize}
		\item 每个人都缺乏``对于自己不熟悉的领域的常识''
		\item 知道该搜索什么, 哪些关键词是与自己手头的问题相关的, 是相当重要并且市场被忽略的.
		\item 实用优先于意识形态, 但不可消灭意识形态.
		\item 优质教程和文档本该存在, 因为它们帮助项目吸引users和collaborators, 你要做的只是找到这些东西,并且跟随它即可.
	\end{itemize}
\end{frame}


%%%%%%%%%%%%%%%%%%%%%%%%%%%%%%%%%%%%%%%%%%%%%%%%%%%%%%%%%%%%%%%%%%%%%%%%%%%%%%%%%%%%%%%%%%%%%%%%%%
%%%%%%%%%%%%%%%%%%%%%%%%%%%%%%%%%%%%%%%%%%%%%%%%%%%%%%%%%%%%%%%%%%%%%%%%%%%%%%%%%%%%%%%%%%%%%%%%%%

\begin{frame}[c]
	\centering \huge{Q\&A}\footnote{
		这次的内容过少,准备不足.效果不佳.\quad 故不会放在github上面.也不会在任何地方发布/分发.\\
		\href{github.com/hehelego/}{Follow me on github}.\\
		We warmly welcome further discussion/error correction/suggestions.
		}
\end{frame}
\end{document}
